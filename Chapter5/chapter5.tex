%!TEX root = ../thesis.tex
%*******************************************************************************
%****************************** Fifth Chapter **********************************
%*******************************************************************************

\chapter{Conclusion}

\section{Summary}

The flanged male orangutan is one of the most extreme forms of sexual dimorphism seen in the primate family. The investigations into the triggers causing the development of these secondary sexual characteristics has been an intriguing avenue of research; however, the costs of their maintenance and their change over time are something so far that has been relatively unexamined. When \citet{Galdikas.1978} noted the presence of a flanged male at Tajung Puting National Park who was "past his prime", given the individual's death the next year, the primary cause was assumed to be old age. However, this assertion was made before we knew the full scope of bimaturism, mainly that unflanged males are not strictly "sub-adult", and are capable of siring children and that development of the flanges can be arrested for decades \citep{Banes.2015cth, Kunz.2023, Prasetyo.2021}. 

Since this first sighting, the appearance of flanged males in this deteriorated state has been observed at many other field sites, with alternative hypotheses suggested for this cause. \citet{Dunkel.2013} suggested the cause may be due to food scarcity, given that orangutans live in forests often characterised by wide variation in fruit availability \citep{Morrogh-Bernard.2011}. When fruit is scarce, orangutans have been documented to have a greatly reduced energy intake and fall into a negative energy balance \citep{Erb.2018}. Although fruit provides significantly greater energetic returns than fall-back foods such as bark or pith, orangutans appear to have developed physiological adaptations to buffer against negative energy balance. Evidence of ketosis in wild orangutans is only seen during periods of low fruit availability, suggesting efficient energy storage as fat during periods of high fruit availability. \citep{Erb.2018, Knott.1998}. Given that flanges morphologically are primarily adipose tissue, under ketosis this adipose tissue may be used for additional energy. Hence, under this hypothesis, the flange serves as an honest signal of energetic balance for the purposes of mating, and the ability of the signaller to buffer against environmental change. 

I hypothesised that one of the causes of this condition is antagonistic social stress. Antagonistic interactions have been shown to increase cortisol levels, which are correlated with inhibited flange development \citep{Thompson.2012}. These interactions can be indirect, as in both unflanged males and flanged males, hearing another flanged male's long call causes an increase in cortisol levels \citep{Prasetyo.2021}. Both of these causes are somewhat secondary and trigger the activation of hormonal pathways for the development of flanges and other associated SSCs. The understanding of the hormonal processes driving bimaturism is murky due to the semi-solitary social structure of orangutans, as well as their arboreal lifestyle making measurement difficult. Current evidence is inconclusive; however, it appears that testosterone is correlated with delayed flange development in captive orangutans \citep{Thompson.2012, Muller.2017}. However, the hormonal profiles of captive orangutans are significantly different to wild populations, for example captive populations experience significantly higher levels of cortisol, and reduced levels of testosterone \citep{Prasetyo.2021}. However, the same hormonal processes that drive initial flange development may also impact the maintenance of these SSCs, and in turn interact with some of the other potential causes, as it has been observed that variation in feeding rates appears to drive differences in cortisol levels in some field sites \citep{Prasetyo.2021}.

And finally there is the original hypothesis of old age, in polygynous and polygynandrous species, senescence is experienced more severely by males compared to females, due to a combination of evolutionary and early life factors \citep{Graves.2007}. Species with high levels of intrasexual competition have a higher risk of mortality, which can lead to a reduction in selection against deleterious mutations that can impact senescence \citep{Williams.1957}. However, these factors are complicated by the individual's history, as early life adversity can impact the onset and severity of senescence \citep{Beirne.2015}. Furthermore, age may act as a synergistic factor that increases the impact of ecological stressors on individuals. Older individuals may have more difficulty accessing or processing food sources, which in turn can affect their health over time. Previous research has indicated correlations between maintained low dominance rank and chronic psycho-social stress in rhesus macaques \citep{Maestripieri.2011}. These impacts then cause an allostatic load on the individual, which can increase the probability of age-related cognitive decline, mortality risk, and cardiovascular disease. These results indicate a synergistic interaction between age and antagonistic interactions; however, social stress in primates depends heavily on the stability of the dominance hierarchies of the chosen species and habitats \citep{Czoty.2009, Mendonça-Furtado.2014, Muller.2004}. 

In this study, my objective was to quantify the size of the flange relative to fixed facial landmarks in two wild populations using pre-existing photographic data sets, to examine whether these changes in relative flange size are correlated with the suggested causative factors, and to delve deeper into correlations between the presumed causative factor and sociality.

Chapter 2 examined whether flanged male orangutans are subject to facial fluctuating asymmetry and whether long-term photographic data sets have sufficient quality to detect this. Additionally, I wished to examine whether the flanges themselves are subject to fluctuating asymmetry. I found that the composite facial relative fluctuating asymmetry scores were higher in Tuanan compared to Suaq Balimbing, which I suggested may be due to the lower overall fruit availability and comparatively poor ecological state of Tuanan. However, flange measurements showed poor repeatability between photos taken on different days, which I suggested may be in part due to significant flange plasticity over time after their development.

Chapter 3 investigated the three potential causes of orangutan flange deformation outlined above, using relative measurements based on the facial landmark measurements collected during Chapter 2, additionally I examined whether there were observed correlations between the change in orangutan flange width and mating and long call behaviour. I found that flange size deterioration is driven primarily by age, and individuals with diminished flanges produce a greater number of long calls per day. I suggest these findings indicate that the maintenance of flange size is not metabolically costly, and variance in flange size after development is likely not due to ketosis, but may be driven by age-related deterioration of the muscle fibres and connective tissue which supports the structure of the flange.

Chapter 4 further delved into the impacts of age on flange males orangutans, utilising a wider behavioural data set from six flanged males who have been observed in the Tuanan field site since their flanging - providing a longitudinal view into how their female associative behaviour and long call composition changes as they grow older. My results indicated that older flanged males tend to have fewer hours of association with females and produce shorter long calls with fewer pulses. These behavioural trends in older flanged males align with observations in other ageing great apes, showing an increase in solitude with age. However, the underlying reasons for this change, whether reduced attractiveness or a self-imposed choice, remain undetermined. Preliminary data also suggest that younger flanged males might exhibit varied responses to certain long call triggers, potentially indicating an effort to establish territorial dominance. However, this behaviour may also be driven by increasing energetic costs for producing long calls in older males.

\section{Implications for orangutan research}

Orangutan research has long focused on understanding the behaviours, physical characteristics, and environmental interactions of these remarkable primates. The findings from this research present some implications for the understanding of both their physical and behavioural attributes and provide direction for avenues of future study, and guidance on how future studies can be refined to take into account some potential shifts in behaviour over the lifetime of flanged males.

Firstly, the discovery that flanged males from the Tuanan field site display a higher asymmetry score compared to the Sumatran site, which I suggest is due to reduced fruit availability and deteriorating ecological health is significant. This highlights the importance of environmental factors in the health and development of orangutans. Additionally, it suggests that the use of historic photographic data sets can be a valuable tool in assessing population health over time. It offers a non-invasive method to track changes in ecological conditions and their subsequent implications on wild animal health. If expanded to use other photographic data sets, this analysis can indirectly examine the role of habitat disturbance, fire, and fluctuations in fruit availability on the growth of individuals. I suggest this technique could be used to examine the impacts of the fires at Tuanan field station on the growth of immature individuals.

Secondly, the tentative confirmation that age appears to be the main driver in the change in flange width over time suggests that the hypothesis that flange deterioration is driven by fruit availability is false. Given that orangutans live in environments typically characterised by wide fluctuations in fruit availability, this suggests orangutans are able to buffer against these shifts. Additionally, as individuals whose flanges are relatively diminished appear to produce long calls less often, this may indicate an aversion to conflict. Examining in further detail the acoustic properties of long calls produced by males with shriveled flanges would shed further light into the acoustic function of flanges. 

Finally, age-related behavioural changes in male orangutans provide a compelling narrative about their social interactions and mating behaviours. The observed trends in sociality in older flanged males, such as reduced association time with females, match patterns seen in other ageing great apes. Understanding whether these behavioural shifts are due to socioemotional selectivity or more simplistic cost-benefit analysis of the risks of association can shed light onto the origins of complex emotional reasoning. The results of this chapter and Chapter 3 suggest that more efforts should be made to control for the impact of age when examining flanged male behaviour, and that the traditional dichotomy of flanged/unflanged behaviour may be a simplification of the true diversity of behaviour shown over the lifetime of these primates.

\section{Conclusion}
Fluctuating asymmetry, as a reflection of developmental instability, has been investigated in the bilateral features of flanged male orangutans across two field sites. Upon analysis, the results of the current research can be summarised as follows:

1. There is a notable difference in facial fluctuating asymmetry between Bornean and Sumatran orangutans, with higher asymmetry scores observed in Bornean orangutans. This finding is attributed to the potentially poorer ecological conditions and reduced fruit availability in the Bornean field site. Moreover, it was determined that the usefulness of historic photographic data sets could be a beneficial method for studying changing ecological conditions and their effects on wild animal health.

2. The study delved into the bimaturism of male orangutans, identifying changes in flange size after its initial development. A critical observation made was the presence of flanged males with deteriorated flanges at multiple sites, commonly referred to as "past-prime". Analysis revealed that such deterioration in flange size is largely driven by age. Furthermore, males with such diminished flanges produced more long calls on a daily basis. The underlying data suggests that maintaining the flange size is not energy-consuming, and variations in flange size post-development may be more closely linked to age-related degradation of the supporting muscle fibres and connective tissue rather than ketosis.

3. Age-related behavioural changes were identified in flanged male orangutans, with changes noted in social behaviour with females and the long call. Analysis from Tuanan field station data indicated that older flanged males, as they age, exhibit reduced hours of association with females and produce shorter long calls with fewer pulses. This is consistent with the observed trends in other great apes, where older individuals tend to become more solitary. However, the exact reason for this change, whether due to reduced attractiveness or a conscious decision, remains ambiguous. Preliminary data hint at potential varied reactions of younger flanged males to specific long call triggers, possibly as an attempt to assert territorial control. This behaviour could also be attributed to the increased energy costs required for longer call production in older males.

In summary, current research has illuminated aspects of facial fluctuating asymmetry, flange deterioration, and age-related behavioural changes in flanged male orangutans. It underscores the significance of understanding the ecological context and the influence of age in shaping the behaviours and physical characteristics of this morph. Future research should delve deeper into the underlying causes and consequences of these findings, improving our understanding of orangutan ecology and evolution.
