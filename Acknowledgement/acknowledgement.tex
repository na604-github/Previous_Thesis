% ************************** Thesis Acknowledgements **************************

\begin{acknowledgements}      

Firstly, I would first like to thank my supervisor, Dr. Sylvain Lemoine, for his assistance throughout the year, particularly for reviewing rough drafts of chapters, providing feedback, and his encouragement to share the results of my second chapter at the 2023 PSGB Spring Conference. I am deeply grateful for the interest in my professional development, his confidence in my ability to complete this research, and for accepting supervision of my project in the first place.

Secondly, I would like to thank Dr. Maria van Noordwijk at the University of Zürich for all her assistance. Primarily for providing access to the long-term data sets of the Tuanan Field Station, without which this project would not have been possible. I am particularly grateful for her feedback on Chapter 2, her encouragement to investigate this topic in the first place and suggesting lines of enquiry.

Thirdly, I would like to thank Dr. Caroline Schuppli and the Suaq Project for providing access to their long-term data set, without which cross-site comparisons would not have been possible. Additionally, I would like to thank her for taking a chance on me to work in Suaq Balimbing as a research assistant and starting me on the academic adventure I find myself in.

I would like to thank Dr. Julia Kunz, for providing early access to her manuscript reviewing the alternate mating strategies of male morphs and for providing her data set of female association with males at Suaq and Tuanan which greatly aided the production of Chapters 3 and 4. 

I would also like to thank my close friend Cameron Pearson for proofreading the entire dissertation before submission while on a delayed train to Manchester.

I gratefully acknowledge the Indonesian State Ministry for Research, Technology and Higher Education (RISTEKDIKTI), the Indonesian Institute of Science (LIPI), the Directorate General of Natural Resources and Ecosystem Conservation–Ministry of Environment \& Forestry of Indonesia (KSDAE-KLHK), the Ministry of Internal affairs, the Nature Conservation Agency of Central Kalimantan (BKSDA), the local governments in Central Kalimantan, the Kapuas Protection Forest Management Unit (KPHL), the Bornean Orangutan Survival Foundation (BOSF), and MAWAS in Palangkaraya as well as the Sumatran orangutan conservation Program (SOCP) and Taman Nasional Gunung Leuser (TNGL) in Medan for their permission and support to conduct this research. I'd also like to thank the Fakultas Biologi Universitas Nasional (UNAS) in Jakarta for their collaboration and support for the Tuanan and Suaq project and in particular Dr. Tatang Mitra Setia. 
 
Thank you to all researchers, students, volunteers, and local field assistants involved in the collection of the standard behavioural data for the long-term databases of Suaq Balimbing and Tuanan. I am also grateful to the local staff at the field sites and the associated offices. 

Finally, thank you to my partner Jessica Smith, for all of her love and support while I spent months measuring orangutan flanges, and for believing in me when I didn't.



\end{acknowledgements}
