% ************************** Thesis Abstract *****************************
% Use `abstract' as an option in the document class to print only the titlepage and the abstract.
\begin{abstract}
Flanged male orangutans exhibit secondary sexual characteristics absent from their unflanged counterparts, the most prominent being their distinctive cheek pads or flanges. Since 1976, there have been observations of flanged males in poor body condition with flanges that appeared shriveled or deformed, a phenomenon tentatively referred to as "past-prime" in the literature. Various hypotheses have been proposed to explain this condition, from age-related degeneration to severe negative energy balances causing ketosis of the fatty content of the flanges. However, so far no attempts have been made to quantify changes in the shape of the flange over time or to prove the causes and social implications of such a condition. In this study, I analysed the relative size and area of the flange of two long-term orangutan field sites using existing photographic data sets and explored the correlations between decreased flange size and social behaviour. The key findings are as follows: 

1) Composite relative fluctuating asymmetry scores were higher in the Bornean field site compared to the Sumatran field site, which I suggest may be due to the lower overall fruit availability and comparatively poor ecological health of the Bornean field site. 

2) Flange measurements showed poor repeatability between photos taken on different days, which I suggest may be in part due to significant flange plasticity over time after their development. 

3) I found that flange width deterioration appears to be driven primarily by age and that individuals with diminished flanges produce more long calls per day. This suggests that the variance in the size of the flange after development is likely not due to ketosis, but may be driven by age-related deterioration of the muscle fibres and connective tissue that supports the structure of the flange. 

4) Older flanged males show significant changes in their long call behaviour, producing shorter long calls with fewer pulses. I suggest that these changes may due to need for recently flanged males to react strongly to local events to establish their presence, and also may hint at the energetic costs of producing long calls becoming more burdensome as individuals grow older. 
\end{abstract}
